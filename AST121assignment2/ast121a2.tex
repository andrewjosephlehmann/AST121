% !TeX program = lualatex

\documentclass{article}
\usepackage{tikz}
\usepackage[utf8]{inputenc}
\setlength{\parindent}{0em} %indents to paragraphs
\setlength{\parskip}{1em} %lineskips after paragraph breaks
\usepackage[margin=1.0in]{geometry} %margins
\usepackage{bbm}
\usepackage{amsmath}
\usepackage{amsthm}
\usepackage{enumerate}
\usepackage{mathtools}
\usepackage{amssymb}
\usepackage{tikz}
\usepackage{tikz-feynman}
\usepackage{amsmath}
\usepackage{physics}
\usetikzlibrary{calc}
\usepackage{feynmp}
\usepackage{feynmp-auto}
\usepackage{breqn}
\usepackage{graphicx}
\newenvironment{amatrix}[1]{%
	\left(\begin{array}{@{}*{#1}{c}|c@{}}
	}{%
	\end{array}\right)
}
\makeatletter
\let\@@span\span
\def\sp@n{\@@span\omit\advance\@multicnt\m@ne}
\makeatother

\renewcommand{\span}{...}


\title{AST121: Assignment 2}
\author{A. Lehmann}
\date{March 2023}

\begin{document}
	\maketitle
	\textit{Assignment 2 iterates upon the concepts introduced in Assignment 1. Hence it is advised to read Assignment 1 before Assignment 2. }
	\tableofcontents
	\newpage 
	\section{Calculation of distance, size, and Hubble Constant for two galaxy clusters}
	\textbar{We observe two arbitrary galaxy clusters, and we find Cluster A is situated at a distance of 50 Mpc, as extrapolated from using Cepheid variables. In addition, we have observed that the galaxies in Cluster B are roughly 64 times dimmer than those in Cluster B. We know the distance of Cluster A. Now how does one uncover how far Cluster B lies from us?}
	\subsection{The Inverse Square Law}
	
	\textbar{The inverse square law describes the relationship between energy density per unit of time and the distance from which we observe a light source, which is almost ubiquitous in many equations describing the workings of the cosmos, such as Newtons universal law of gravitation, which states that the strength of a gravitational field decreases with the square of the distance.}
	\begin{gather*}
	F_g = \frac{GMm}{r^2} \implies F_g \propto \frac{1}{r^2}
	\end{gather*}

	\textbar{One can come about the inverse square law when pondering the geometry of light dispersing out in a three-dimensional space. When light radiates from a point source, it spreads out evenly in all directions, forming a sphere of light. As the sphere expands, the same amount of light is distributed over a larger and larger surface area. Therefore, the light's intensity decreases, with the square of the area, as it travels further away from its light source.}
	\begin{center}
		\textit{Surface Area of Sphere = $4\pi r^2, $ where $r$ is the radius}
	\end{center}

\textbar{Now, we can utilize Maxwell's first equation in integral form to derive the inverse square law.}
\begin{center}
\textit{Maxwell's first equation, (refered to as Gauss's law):}
$\boxed{\int E \cdot d\vec{a}  = \frac{Q_{encl}}{\epsilon_0}}
$

\end{center} 
\textit{We may rewrite for the derivation of inverse square law: }
\begin{gather}
\iint_{\mathcal{S}_c} E \cdot d\vec{a} = 4\pi \iiint_V \rho dV 
\end{gather}
\begin{gather}
E(r) \iint_{\mathcal{S}_c} d\vec{a} = 4\pi r^2 E(r)
\end{gather}
\begin{gather}
4\pi r^2 E(r) = 4\pi e
\end{gather}
\begin{center}
	$\boxed{E(r) = \frac{e}{r^2}}$
	
\end{center}


%%%ADD MORE EXPLAINTIONS TO HOW I DERIVIED THIS BEAST%%%

\textbar{From our understanding of inverse square law, we can conclude:}
	\begin{gather*}
	I = \frac{L}{4\pi r^2} \implies I \propto \frac{1}{r^2}
	\end{gather*}
\textbar{Now we can solve for the distance to Cluster B. We know the distance to Cluster A is 50 Mpc, and the galaxies in Cluster B are 64 times fainter than A. With inverse square law, we may write this as:}
\begin{gather*}
	I \propto \frac{1}{r^2} \implies \Delta I = \frac{d_{B}^2}{d_{A}^2}
\end{gather*}
\begin{gather*}
\Delta I \times d^2_A = d^2_B
\end{gather*}
\begin{center}
$\boxed{ d_B = \sqrt{\Delta I \times d^2_A} = \sqrt{64 \times 50^2Mpc} = 400Mpc}
$
\end{center}
\textbar{Using the inverse square law, we deduced that Cluster B is approximately 400 Mpc away from us.}	

\subsection{Cosmological dimming and its influence on the apparent size of distant objects}

\textbar{If one were to probe the sky for these two cosmic objects, one would notice an apparent delta in brightness and size when comparing Cluster A and B. Cluster B would appear noticeably smaller and fainter. We may explain by the inverse square law, for when the light from the galaxies in Cluster has travelled a greater distance throughout space, causing it to become dimmer. As a result, Cluster B takes up a smaller subset of the sky compared to Cluster A. However, if we consider the luminosity of these two clusters, we observe them to be approximately the same.
	
The connection between the size of an object and its luminosity can vary depending on the class of cosmic object we are studying. Commonly, larger objects tend to be more luminous than smaller objects of the same kind. The relation exists because larger objects typically have more mass and can generate more energy through nuclear fusion, producing the electromagnetic radiation we observe as luminosity.
	
Additionally, it is worth noting that a galaxy cluster's apparent size and brightness are not always directly proportional to its actual size and luminosity. A cluster's observed size and brightness can also be altered by factors such as its distance from us, the presence of foreground or background objects, and the effects of gravitational lensing. To find the average observed size, let us assume the two galaxy clusters are of equal length. With that, we can emanate how much larger Cluster A is in the sky compared with Cluster B.}

\textit{Demonstrating our assumption may be true, considering that we approximated the distance of Cluster B to us to be $400Mpc$:}
\begin{center}
	\textit{Assume,} $L \propto \mathcal{S}$
	
	$\frac{L_A}{L_B} = \Delta I\frac{d_{A}^2}{d_{B}^2} = 1 
	$
	
	$\boxed{\mathcal{A}_{\mathcal{S}} = \frac{d_B}{d_A} = \frac{50Mpc}{400Mpc} = \frac{1}{8}}
	$
\end{center}
\textit{With this, we deduced that Cluster A amasses eight times more area in the sky than Cluster B. }



\subsection{Calculating the Hubble Constant from redshift and distance}
\textbar{We observe the redshift of Cluster B to be $z_B = 0.1$. With this information, we must relate the redshift to our equations with relativistic redshift to find the Hubble constant.}
\begin{center}
	\textit{Let us define Hubble's constant to be,  }$H_0 = \frac{v}{d}$, and relatistic redshift to be, $z = \frac{v}{c}. $
\end{center}
	\textit{Thus,}
\begin{gather*}
	H_0 = \frac{v}{d} = \frac{cz}{d}
\end{gather*}
\begin{center}
\boxed{H_0 = \frac{cz_B}{d_B} = \frac{(3\times10^6km/s)(0.1)}{400Mpc} = 75kms^{-1}/Mpc}
\end{center}
\textbar{Now we know that the Hubble constant, we can find the redshift of Cluster A. So let us solve for the redshift with our derivation of the relation of Hubbles constant to relativistic redshift.}
\begin{gather*}
	H_0 = \frac{cz}{d} \implies z = \frac{H_0d}{c}
\end{gather*}
\begin{center}
	$\boxed{z_A = \frac{H_0d_A}{c} = \frac{(7.5\times10^4ms^{-1}/Mpc)(50Mpc)}{3\times10^{8}ms^{-1}}=\frac{1}{80}=0.125}
	$
\end{center}

	\section{Deriving Special Relativity}
	
	
	
	
	
	
	
	
	\section{A Solution to Einsteins Field Equations}
\end{document}
