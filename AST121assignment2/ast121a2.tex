% !TeX program = lualatex

\documentclass{article}
\usepackage{tikz}
\usepackage[utf8]{inputenc}
\setlength{\parindent}{0em} %indents to paragraphs
\setlength{\parskip}{1em} %lineskips after paragraph breaks
\usepackage[margin=1.0in]{geometry} %margins
\usepackage{bbm}
\usepackage{amsmath}
\usepackage{amsthm}
\usepackage{enumerate}
\usepackage{mathtools}
\usepackage{amssymb}
\usepackage{tikz}
\usepackage{tikz-feynman}
\usepackage{amsmath}
\usepackage{physics}
\usetikzlibrary{calc}
\usepackage{feynmp}
\usepackage{feynmp-auto}
\usepackage{breqn}
\usepackage{graphicx}
\newenvironment{amatrix}[1]{%
	\left(\begin{array}{@{}*{#1}{c}|c@{}}
	}{%
	\end{array}\right)
}
\makeatletter
\let\@@span\span
\def\sp@n{\@@span\omit\advance\@multicnt\m@ne}
\makeatother

\renewcommand{\span}{...}


\title{AST121: Assignment 2}
\author{A. Lehmann}
\date{March 2023}

\begin{document}
	\maketitle
	\textit{Assignment 2 iterates upon the concepts introduced in Assignment 1. Hence it is advised to read Assignment 1 before Assignment 2. }
	\tableofcontents
	\newpage 
	\section{Calculation of distance, size, and Hubble Constant for two galaxy clusters}
	\textbar{We observe two arbitrary galaxy clusters, and we find Cluster A is situated at a distance of 50 Mpc, as extrapolated from using Cepheid variables. In addition, we have observed that the galaxies in Cluster B are roughly 64 times dimmer than those in Cluster B. We know the distance of Cluster A. Now how does one uncover how far Cluster B lies from us?}
	
	
	
	
	
	\section{Deriving Special Relativity}
	
	
	
	
	
	
	
	
	\section{A Solution to Einsteins Field Equations}
\end{document}
