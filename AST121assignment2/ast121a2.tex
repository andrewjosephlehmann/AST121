% !TeX program = lualatex

\documentclass{article}
\usepackage{tikz}
\usepackage[utf8]{inputenc}
\setlength{\parindent}{0em} %indents to paragraphs
\setlength{\parskip}{1em} %lineskips after paragraph breaks
\usepackage[margin=1.0in]{geometry} %margins
\usepackage{bbm}
\usepackage{amsmath}
\usepackage{amsthm}
\usepackage{enumerate}
\usepackage{mathtools}
\usepackage{amssymb}
\usepackage{tikz}
\usepackage{tikz-feynman}
\usepackage{amsmath}
\usepackage{physics}
\usetikzlibrary{calc}
\usepackage{feynmp}
\usepackage{feynmp-auto}
\usepackage{breqn}
\usepackage{graphicx}
\newenvironment{amatrix}[1]{%
	\left(\begin{array}{@{}*{#1}{c}|c@{}}
	}{%
	\end{array}\right)
}
\makeatletter
\let\@@span\span
\def\sp@n{\@@span\omit\advance\@multicnt\m@ne}
\makeatother

\renewcommand{\span}{...}


\title{AST121: Assignment 2}
\author{A. Lehmann}
\date{March 2023}

\begin{document}
	\maketitle
	\textit{Assignment 2 iterates upon the concepts introduced in Assignment 1. Hence it is advised to read Assignment 1 before Assignment 2. }
	\tableofcontents
	\newpage 
	\section{Calculation of distance, size, and Hubble Constant for two galaxy clusters}
	\textbar{We observe two arbitrary galaxy clusters, and we find Cluster A is situated at a distance of 50 Mpc, as extrapolated from using Cepheid variables. In addition, we have observed that the galaxies in Cluster B are roughly 64 times dimmer than those in Cluster B. We know the distance of Cluster A. Now how does one uncover how far Cluster B lies from us?}
	\subsection{The Inverse Square Law}
	
	\textbar{The inverse square law describes the relationship between energy density per unit of time and the distance from which we observe a light source, which is almost ubiquitous in many equations describing the workings of the cosmos, such as Newtons universal law of gravitation, which states that the strength of a gravitational field decreases with the square of the distance.}
	\begin{gather*}
	F_g = \frac{GMm}{r^2} \implies F_g \propto \frac{1}{r^2}
	\end{gather*}
	\textbar{One can come about the inverse square law when pondering the geometry of light dispersing out in a three-dimensional space. When light radiates from a point source, it spreads out evenly in all directions, forming a sphere of light. As the sphere expands, the same amount of light is distributed over a larger and larger surface area. Therefore, the light's intensity decreases, with the square of the area, as it travels further away from its light source.}
	\begin{center}
		\textit{Area of Sphere = $4\pi r^2, $ where $r$ is the radius}
	\end{center}
\textbar{From our understanding of inverse square law, we can conclude:}
	\begin{gather*}
	I = \frac{L}{4\pi r^2} \implies I \propto \frac{1}{r^2}
	\end{gather*}
\textbar{Now we can solve for the distance to Cluster B. We know the distance to Cluster A is 50 Mpc, and the galaxies in Cluster B are 64 times fainter than A. With inverse square law, we may write this as:}
\begin{gather*}
	I \propto \frac{1}{r^2} \implies \Delta I = \frac{d_{B}^2}{d_{A}^2}
\end{gather*}
\begin{gather*}
\Delta I \times d^2_A = d^2_B
\end{gather*}
\begin{center}
$\boxed{ d_B = \sqrt{\Delta I \times d^2_A} = \sqrt{64 \times 50^2Mpc} = 400Mpc}
$
\end{center}
	
	\section{Deriving Special Relativity}
	
	
	
	
	
	
	
	
	\section{A Solution to Einsteins Field Equations}
\end{document}
