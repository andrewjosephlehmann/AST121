% !TeX program = lualatex

\documentclass{article}
\usepackage{tikz}
\usepackage[utf8]{inputenc}
\setlength{\parindent}{0em} %indents to paragraphs
\setlength{\parskip}{1em} %lineskips after paragraph breaks
\usepackage[margin=1.0in]{geometry} %margins
\usepackage{bbm}
\usepackage{amsmath}
\usepackage{amsthm}
\usepackage{enumerate}
\usepackage{mathtools}
\usepackage{amssymb}
\usepackage{tikz}
\usepackage{tikz-feynman}
\usepackage{amsmath}
\usepackage{physics}
\usetikzlibrary{calc}
\usepackage{feynmp}
\usepackage{feynmp-auto}
\usepackage{breqn}
\usepackage{graphicx}
\newenvironment{amatrix}[1]{%
	\left(\begin{array}{@{}*{#1}{c}|c@{}}
	}{%
	\end{array}\right)
}
\makeatletter
\let\@@span\span
\def\sp@n{\@@span\omit\advance\@multicnt\m@ne}
\makeatother

\renewcommand{\span}{...}
\usepackage{biblatex}





\title{AST121: Assignment 2}
\author{A. Lehmann}
\date{March 2023}

\begin{document}
	\maketitle

	
	\newpage
	\tableofcontents
	\newpage 
	\section{Calculation of distance, size, and Hubble Constant for two galaxy clusters}
	\textbar{We observe two arbitrary galaxy clusters, and we find Cluster A is situated at a distance of 50 Mpc, as extrapolated from using Cepheid variables. In addition, we have observed that the galaxies in Cluster B are roughly 64 times dimmer than those in Cluster B. We know the distance of Cluster A. Now how does one uncover how far Cluster B lies from us?}
	\subsection{The Inverse Square Law}
	
	\textbar{The inverse square law describes the relationship between energy density per unit of time and the distance from which we observe a light source, which is almost ubiquitous in many equations describing the workings of the cosmos, such as Newtons universal law of gravitation, which states that the strength of a gravitational field decreases with the square of the distance.}
	\begin{gather*}
	F_g = \frac{GMm}{r^2} \implies F_g \propto \frac{1}{r^2}
	\end{gather*}

	\textbar{One can come about the inverse square law when pondering the geometry of light dispersing out in a three-dimensional space. When light radiates from a point source, it spreads out evenly in all directions, forming a sphere of light. As the sphere expands, the same amount of light is distributed over a larger and larger surface area. Therefore, the light's intensity decreases, with the square of the area, as it travels further away from its light source.}
	\begin{center}
		\textit{Surface Area of Sphere = $4\pi r^2, $ where $r$ is the radius}
	\end{center}

\textbar{Now, we can utilize Maxwell's first equation in integral form to derive the inverse square law.}
\begin{center}
\textit{Maxwell's first equation, (refered to as Gauss's law):}
$\boxed{\int E \cdot d\vec{a}  = \frac{Q_{encl}}{\epsilon_0}}
$

\end{center} 
\textit{We may rewrite for the derivation of inverse square law: }
\begin{gather}
\iint_{\mathcal{S}_c} E \cdot d\vec{a} = 4\pi \iiint_V \rho dV 
\end{gather}
\begin{gather}
E(r) \iint_{\mathcal{S}_c} d\vec{a} = 4\pi r^2 E(r)
\end{gather}
\begin{gather}
4\pi r^2 E(r) = 4\pi e
\end{gather}
\begin{center}
	$\boxed{E(r) = \frac{e}{r^2}}$
	
\end{center}


%%%ADD MORE EXPLAINTIONS TO HOW I DERIVIED THIS BEAST%%%

\textbar{From our understanding of inverse square law, we can conclude:}
	\begin{gather*}
	I = \frac{L}{4\pi r^2} \implies I \propto \frac{1}{r^2}
	\end{gather*}
\textbar{Now we can solve for the distance to Cluster B. We know the distance to Cluster A is 50 Mpc, and the galaxies in Cluster B are 64 times fainter than A. With inverse square law, we may write this as:}
\begin{gather*}
	I \propto \frac{1}{r^2} \implies \Delta I = \frac{d_{B}^2}{d_{A}^2}
\end{gather*}
\begin{gather*}
\Delta I \times d^2_A = d^2_B
\end{gather*}
\begin{center}
$\boxed{ d_B = \sqrt{\Delta I \times d^2_A} = \sqrt{64 \times 50^2Mpc} = 400Mpc}
$
\end{center}
\textbar{Using the inverse square law, we deduced that Cluster B is approximately 400 Mpc away from us.}	

\subsection{Cosmological dimming and its influence on the apparent size of distant objects}

\textbar{If one were to probe the sky for these two cosmic objects, one would notice an apparent delta in brightness and size when comparing Cluster A and B. Cluster B would appear noticeably smaller and fainter. We may explain by the inverse square law, for when the light from the galaxies in Cluster has travelled a greater distance throughout space, causing it to become dimmer. As a result, Cluster B takes up a smaller subset of the sky compared to Cluster A. However, if we consider the luminosity of these two clusters, we observe them to be approximately the same.
	
The connection between the size of an object and its luminosity can vary depending on the class of cosmic object we are studying. Commonly, larger objects tend to be more luminous than smaller objects of the same kind. The relation exists because larger objects typically have more mass and can generate more energy through nuclear fusion, producing the electromagnetic radiation we observe as luminosity.
	
Additionally, it is worth noting that a galaxy cluster's apparent size and brightness are not always directly proportional to its actual size and luminosity. A cluster's observed size and brightness can also be altered by factors such as its distance from us, the presence of foreground or background objects, and the effects of gravitational lensing. To find the average observed size, let us assume the two galaxy clusters are of equal length. With that, we can emanate how much smaller Cluster B is in the sky compared with Cluster A.}

\textit{Demonstrating our assumption may be true, considering that we approximated the distance of Cluster B to us to be $400Mpc$:}
\begin{center}
	\textit{Assume,} $L \propto \mathcal{S}$
	
	$\frac{L_A}{L_B} = \Delta I\frac{d_{A}^2}{d_{B}^2} = 1 
	$
	
	$\boxed{\mathcal{A}_{\mathcal{S}} = \frac{d_B}{d_A} = \frac{50Mpc}{400Mpc} = \frac{1}{8}}
	$
\end{center}
\textit{With this, we deduced that Cluster A amasses eight times more area in the sky than Cluster B. }



\subsection{Calculating the Hubble Constant from redshift and distance}
\textbar{We observe the redshift of Cluster B to be $z_B = 0.1$. With this information, we must relate the redshift to our equations with relativistic redshift to find the Hubble constant.}
\begin{center}
	\textit{Let us define Hubble's constant to be,  }$H_0 = \frac{v}{d}$, and redshift to be, $z = \frac{v}{c}. $
\end{center}
	\textit{Thus,}
\begin{gather*}
	H_0 = \frac{v}{d} = \frac{cz}{d}
\end{gather*}
\begin{center}
\boxed{H_0 = \frac{cz_B}{d_B} = \frac{(3\times10^5km/s)(0.1)}{400Mpc} = 75kms^{-1}/Mpc}
\end{center}
\textbar{Now we know that the Hubble constant, we can find the redshift of Cluster A. So let us solve for the redshift with our derivation of the relation of Hubbles constant to relativistic redshift.}
\begin{gather*}
	H_0 = \frac{cz}{d} \implies z = \frac{H_0d}{c}
\end{gather*}
\begin{center}
	$\boxed{z_A = \frac{H_0d_A}{c} = \frac{(7.5\times10^4ms^{-1}/Mpc)(50Mpc)}{3\times10^{8}ms^{-1}}=\frac{1}{80}=0.125}
	$
\end{center}

	\section{Special Relativity}
	
\textbar{Special Relativity, or what Einstein initially called the electrodynamics of moving bodies, is a theory that unites electromagnetism and non-accelerating classical mechanical systems. Special Relativity is particularly useful when describing how the motion of objects appears in different reference frames. However, before delving into its derivation, we must be conscious of the two postulates made for Special Relativity. The first postulate is that all physical laws are the same for all inertia frames of reference, regardless of motion. The second postulate is that the speed of light is constant irrespective of where you are in the vacuum of space. }

\subsubsection{Derivation of the Lorentz Transformation in
	Einstein’s Theory of Special Relativity}

	\begin{figure}[h]
	\begin{center}
		
		
		\begin{tikzpicture}
			\draw[very thick,black](-5.5,5) -- (-4.5,5);
			\draw[very thick, purple, -] (-5,0) -- node[left]{$x = \frac{ct}{2}$}   (-5,5);
			\draw[very thick, purple, <->] (-5,1) --   (-5,4);
			\draw[very thick,black](-5.5,0) -- (-4.5,0);
			
		\end{tikzpicture}
	\end{center}
\end{figure}

Suppose we are in a hypothetical spacecraft moving at relativistic speed. Our hypothetical spacecraft contains a small light clock visible from the earth, which aids us in visualizing the differences observed from within and outside the spacecraft. From within the spaceship, we find a light source and two mirrors reflecting the light back and forth. Therefore, according to the observer in the rocket, the light travels in a straight line. 

	\begin{figure}[h]
	\begin{center}	
		\begin{tikzpicture}
			\draw[very thick,black](-4.5,5) -- (-3.5,5);
			\draw[very thick, purple, ->] (-5,0) -- node[left]{}   (-4,5);
			\draw[very thick,black](-5.5,0) -- (-4.5,0);
			\draw[very thick, purple, ->] (-4,5) --  (-3,0);

			\draw[very thick,black](-2.5,0) -- (-3.5,0);
		\end{tikzpicture}	
	\end{center}
\end{figure}

However, from an external observer, let us say we find a contradiction. The light appears not to be travelling in a straight path. With this, we can divide our triangle in two to form a right-angle triangle. We can subsequently use the Pythagorean theorem:

\begin{gather*}
	x^2 =  \left(\frac{ct'}{2}\right)^2 - \left(\frac{vt'}{2}\right)^2
\end{gather*}
\begin{gather*}
x^2 = \frac{c^2{t'}^2}{4} - \frac{v^2{t'}^2}{4}
\end{gather*}
\textit{We know in the other reference frame $x = \frac{ct}{2}$. Thus:}
\begin{gather*}
	\left(\frac{ct}{2}\right)^2 = \frac{c^2{t'}^2 - v^2{t'}^2}{4} 
\end{gather*}
\begin{gather*}
	\frac{c^2t^2}{4} = \frac{{t'}^2(c^2 - v^2)}{4} 
\end{gather*}
\begin{gather*}
	c^2t^2 = {t'}^2(c^2-v^2)
\end{gather*}
\begin{gather*}
	\frac{c^2t^2}{c^2 - v^2} = {t'}^2
\end{gather*}
\begin{gather*}
	{t'}^2 = \frac{t^2}{1 - \frac{v^2}{c^2}}
\end{gather*}
\begin{gather*}
\frac{{t'}}{t} = \sqrt{\frac{1}{1 - \frac{v^2}{c^2}}}
\end{gather*}
\textit{We know $\gamma = \frac{t'}{t}$, thus:}
\begin{center}
	$\boxed{ \gamma = {\frac{1}{\sqrt{1 - \frac{v^2}{c^2}}}}}
	$
\end{center}

\subsubsection{Derivation of relativistic doppler shift}


How does this impact the wavelength of photons emitted by objects at such speeds? For this, we must derive the relativistic doppler shift. 

\textit{Let, $\beta = \frac{v^2}{c^2}$}

\begin{gather*}
\Lambda_{00} = \gamma \implies \Lambda_{0i} = \Lambda_{i0} = -\gamma\beta_i \implies 	\Lambda_{ij} = (\gamma - 1)\left(\frac{\beta_i\beta_j}{\beta^2}\right) + \delta_{ij}
\end{gather*}
\begin{gather*}
\Lambda = \begin{bmatrix} \gamma & -\gamma\beta & 0 & 0 \\
	-\gamma\beta & \gamma & 0 & 0 \\
	0 & 0 & 1 & 0 \\
	0 & 0 & 0 & 1
\end{bmatrix} \implies 	x = \begin{bmatrix} hf_0 \\ -hf_0 \\ 0 \\ 0
\end{bmatrix} x' = \begin{bmatrix} hf_s \\ -hf_s \\ 0 \\ 0 \end{bmatrix}
\end{gather*}
\begin{gather*}
x' = \Lambda x \implies  \begin{bmatrix} hf_s \\ -hf_s \\ 0 \\ 0 \end{bmatrix} = \begin{bmatrix} \gamma & -\gamma\beta & 0 & 0 \\
	-\gamma\beta & \gamma & 0 & 0 \\
	0 & 0 & 1 & 0 \\
	0 & 0 & 0 & 1
\end{bmatrix} \begin{bmatrix} hf_0 \\ -hf_0 \\ 0 \\ 0
\end{bmatrix} 
\end{gather*}
\begin{gather*}
\begin{bmatrix} \gamma h f_0 + \gamma\beta h f_0 \\ -(\gamma\beta hf_0 + \gamma h f_0) \\ 0 \\ 0\end{bmatrix} = \gamma(\beta + 1) \begin{bmatrix} hf_0 \\ -hf_0 \\ 0 \\ 0 \end{bmatrix}
\end{gather*}
\textit{With this, we may rewrite this expression as:}
\begin{gather*}
\frac{1}{f_0} = \gamma\frac{1}{f_s} + \gamma\beta\frac{1}{f_s} \implies \frac{f_s}{f_0} = \gamma + \gamma\beta
\end{gather*}
\begin{gather*}
\frac{f_s}{f_0} = \gamma(1+\beta) = \frac{1}{\sqrt{1-\beta^2}}(1+\beta)
\end{gather*}
\begin{gather*}
	\frac{f_s}{f_0} = \frac{1+\beta}{\sqrt{1-\beta^2}} = \frac{1+\beta}{\sqrt{(1-\beta)(1+\beta)}} = \frac{\sqrt{1+\beta}}{\sqrt{1-\beta}}
\end{gather*}
\begin{center}
	$\boxed{\frac{f_s}{f_0} = \sqrt{\frac{1+\beta}{1-\beta}}}
	$
\end{center}
\textit{We can also correspond this to the wavelength in the relativistic longitudinal doppler effect: }
\begin{center}
	$\boxed{\frac{\delta\lambda}{\lambda_0} = \sqrt{\frac{1+\beta}{1-\beta}}}
$
\end{center}
\newpage
It is also important to note that at higher values of $\beta$, the difference between Classical Doppler Shift and Relativistic Doppler becomes pronounced.
\begin{center}

	\includegraphics[width=0.5\linewidth]{../../../../ReldopvsCdop}

\end{center}



To understand these derivations further, let us think of scenarios where we observe objects moving at relativistic speeds. Therefore we shall comprise various situations where this may be useful.

\subsection{Relativistic Doppler Shift Calculation for Two Approaching Spaceships}

Two spaceships (A and B) approach you at 9/10ths the speed of light (0.9c) from opposite directions. They send out radio messages. What speed does one measure for the radio waves from A and B? 


Considering the second postulate of special relativity, we assume the speed of light is constant irrespective of where you are in the vacuum of space. Therefore, since radio waves are electromagnetic radiation, that implies that the radiowaves travels at the speed of light in a vacuum. Therefore, the speed at which we measure the radiowaves emitted by Rocket A and B are the same and are travelling at the speed of light $c$. 


Now what would be the speed of the radio waves from A as measured by B? Since the speed of light is constant for each observer, we may also conclude that the rockets measure each other's radiowaves to be the speed of light $c$.


What is the speed that Spaceship B measures your motion? Considering that each observer is in an inertial frame of reference, each object in this scenario seems to be at rest while the other objects are moving. Therefore, since Spaceship B appears to be approaching the earth at $0.9c$, we know that according to the rocket, it seems the earth is moving towards it at $0.9c$. Thus, Spaceship B measures my motion to be $0.9c$.

\subsubsection{How fast does Spaceship B observe Spaceship A to be moving?}

Utilizing our derivation of the lorentz transformation, we may rewrite the expression in terms of velocity addition: 
\begin{gather*}
\gamma = {\frac{1}{\sqrt{1 - \frac{v^2}{c^2}}}} \implies {v'}_A = \frac{v_A - v_B}{1 - \frac{(v_A)(v_B)}{c^2}} 
\end{gather*}
\begin{center}
	$\boxed{{v'}_A = \frac{v_A - v_B}{1 - \frac{(v_A)(v_B)}{c^2}} =  \frac{0.9c - (-0.9c)}{1 - \frac{(0.9c)(-0.9c)}{c^2}} = 0.9945c}
	$
\end{center}

Therefore, Spaceship B observes Spaceship A to be moving at $0.9945c$.
\subsubsection{Doppler Shift of Radiowaves emitted from relativistic speeds }


Assume the radio messages are emitted from the spaceships with a rest wavelength of 1cm. What is the Doppler Shift that you observe in message A? 

Previously we concluded that Spaceship A was travelling towards us at a speed of 0.9c. With our derivation of the doppler shift, we can find the observed wavelength of the radio wave.
\begin{gather*}
	\frac{\delta\lambda}{\lambda_0} = \sqrt{\frac{1+\beta}{1-\beta}} \implies 	\delta\lambda= \lambda_0\sqrt{\frac{1+\beta}{1-\beta}}
\end{gather*}

With this information, we can deduce that the observed wavelength:
%%%%%% REVIEW THIS BEFORE SUBMITING %%%

\begin{center}
	$
	\boxed{\delta\lambda = \lambda_0\sqrt{\frac{1+\beta}{1-\beta}} =(1.0\times10^{-2}m) \sqrt{\frac{1+\frac{-0.9c}{c}}{1-\frac{-0.9c}{c}}}=2.3\times10^{-3}m}
	$

\end{center}
Since Spaceship B travels towards us at the same speed as Spaceship A, this implies that the observed wavelength of the radiowaves from Spaceship B will be equivalent to that of A. 

What is the Doppler Shift that A would observe in Spaceship B? 

\begin{center}
	$
	\boxed{\delta\lambda = \lambda_0\sqrt{\frac{1+\beta}{1-\beta}} =(1.0\times10^{-2}m) \sqrt{\frac{1+\frac{-0.9945c}{c}}{1-\frac{-0.9945c}{c}}}=5.3\times10^{-4}m}
	$
	
\end{center}




	\section{Quasars and the implications of high redshifts}
	
\textbar{When peering into the center of galaxies, what is often encountered are massive objects known as Quasars. These are shining objects typically found at the centers of galaxies. Initially known as quasi-stellar objects, later, it was realized that their light is emitted from an accretion disk of gas orbiting in close proximity to supermassive black holes. On occurrence, the redshifts of quasars can be as high as four or greater. Allow us to explore this concept and further comprehend how it contributes to our understanding of the workings of quasars and the expansion of the cosmos at large. 
	
Since we know a subset of Quasars has a redshift of four or more. We can use the equations for Schwarzschild radius and its relation to redshift to find how close the accretion disk must sit relative to the blackhole to emit electromagnetic spectra with a redshift of four or more. 
}

\textit{Let,}
\begin{gather*}
	z = {\left(1-\frac{2GM}{rc^2}\right)^{\frac{-1}{2}} - 1}
\end{gather*}
\textit{and,}
\begin{gather*}
	\mathcal{R}_s = \frac{2GM}{c^2}
\end{gather*}
Now let us rearrange:
\begin{gather*}
	z+1 =  {\left(1-\frac{2GM}{rc^2}\right)^{\frac{-1}{2}} \implies \frac{1}{(z+1)^2} = 1 - \frac{2GM}{rc^2} \implies \frac{2GM}{rc^2} = 1 - \frac{1}{(z+1)^2} }
\end{gather*}
\begin{gather*}
	\frac{2GM}{rc^2} = 1 - \frac{1}{(z+1)^2} \implies \frac{2GM}{c^2} = \left(1-\frac{1}{(z+1)^2}\right)r
\end{gather*}
\textit{Since, $\mathcal{R}_s = \frac{2GM}{c^2}$, then:}
\begin{gather*}
	\mathcal{R}_s =  \left(\frac{1}{1-\frac{1}{(z+1)^2}}\right)r \implies \mathcal{R}_s\left(\frac{1}{1-\frac{1}{(z+1)^2}}\right) = r \implies \frac{r}{\mathcal{R}_s} = \left(\frac{1}{1-\frac{1}{(z+1)^2}}\right) 
\end{gather*}

Let $z = 4$
\begin{center}
	$\boxed{\frac{r}{\mathcal{R}_s} = \left(\frac{1}{1-\frac{1}{(z+1)^2}}\right) = \left(\frac{1}{1-\frac{1}{(4+1)^2}}\right) = \frac{25}{24}}
	$
\end{center}


\textbar{I find this to be a surprising result mainly since this implies that the emission of the photon lay beyond the photon sphere of the black hole. Hence, the photons that are being emitted beyond a point where there is not enough angular momentum for them to escape the gravitational pull. In other words, the photons would have to orbit the black hole faster than the speed of light to escape. For a Schwazchild black hole, the photon sphere lies at $\frac{3r_s}{2}$.
\begin{center}
	
	\includegraphics[width=0.5\linewidth]{../../../../RedshiftSchwarz}

	\label{fig:redshiftschwarz}
\end{center}

With this graph in mind, let us find the maximum redshift of a photon for Schwarzchild black hole. To uncover this, we take the inverse of the equation for Schwarzschild radius: 

\begin{gather*}
	\mathcal{R}_s = \left(1-\frac{1}{(z+1)^2}\right)r \implies \frac{\mathcal{R}_s}{r} = 1 - \frac{1}{(z+1)^2} \implies \frac{\mathcal{R}_sr(z+1)^2}{r} - \frac{r(z+1)^2}{\mathcal{R}_s} = -\frac{r(z+1)^2}{\mathcal{R}_s(z+1)^2}
\end{gather*}
\begin{gather*}
	(z+1)^2 - \frac{r}{\mathcal{R}_s(z+1)^2} = \frac{-r}{\mathcal{R}_s} \implies (z+1)^2 - \frac{r}{\mathcal{R}_s(z+1)^2} + \frac{r}{\mathcal{R}_s} = 0 
\end{gather*}
Using the quadratic equation, we may rewrite the expression as: 

\begin{center}
$\boxed{z = - \frac{\frac{r}{\mathcal{R}_s}-1-\sqrt{\frac{r}{\mathcal{R}_s}(\frac{r}{\mathcal{R}_s}-1)}}{\frac{r}{\mathcal{R}_s}-1}}
$
\end{center}
Now we can insert the radius of photon sphere to find it's maximum redshift of photons emitted from blackhole: 
\begin{center}
$\boxed{z = - \frac{\frac{3}{2}-1-\sqrt{\frac{3}{2}(\frac{3}{2}-1)}}{\frac{3}{2}-1} = \frac{21}{29} \approx 0.724}
$
\end{center}


These results allude that the redshift we observe is not solely from the gravitational redshift, but rather, the quasar is moving away from us at high velocities. Let us use our derivation of the relativistic doppler shift and relay it to the redshift:

\begin{gather*}
	\frac{\delta\lambda}{\lambda_0} = \sqrt{\frac{1+\beta}{1-\beta}} \implies z + 1 = \sqrt{\frac{1+\beta}{1-\beta}} = \sqrt{\frac{c+v}{c-v}}
\end{gather*}

\begin{center}
	$\boxed{z= \sqrt{\frac{c+v}{c-v}} - 1}
	$
\end{center}

Let us rewrite this expression in terms of velocity: 
%v = \frac{z(z+2)}{z^2+2z+2}
\begin{gather*}
z = \sqrt{\frac{c+v}{c-v}} - 1 \implies v = \left(\frac{z(z+2)}{z^2+2z+2}\right)c
\end{gather*}

\begin{center}
	$\boxed{v = \left(\frac{z(z+2)}{z(z+2)+2}\right)c}
	$
\end{center}
\begin{center}
	
	\includegraphics[width=0.5\linewidth]{../../../../RecessionVel}
	\label{fig:recessionvel}
\end{center}
Since we observed the redshift of the quasar to be $z=4$, then:

\begin{center}
	$\boxed{v = \left(\frac{z(z+2)}{z(z+2)+2}\right)c = \left(\frac{4(4+2)}{4(4+2)+2}\right)c = \frac{12}{13}c \approx 0.923c}
	$
\end{center}

With this simple calculation, we have determined that this quasar is moving away from us at a relativistic speed. Furthermore, we often observe many Quasars from redshifts of $z=0.05$ to $z=8$, and thus this phenomenon is conceivable to be an artifact of an expanding universe. 

\section{The Scale Factor}

\begin{gather*}
a \propto t^{\frac{2}{3}} \implies a = kt^{\frac{2}{3}} \implies \frac{a_0}{a_1} = \frac{t_0^{\frac{2}{3}}}{t_1^{\frac{2}{3}}}
\end{gather*}
\textit{We know $a_0=1$, thus:}
\begin{gather*}
\frac{1}{a_1} = \frac{t_0^{\frac{2}{3}}}{t_1^{\frac{2}{3}}} \implies a_1 = \frac{t_1^{\frac{2}{3}}}{t_0^{\frac{2}{3}}}
\end{gather*}
\textit{The relation of scale factor and redshift is $a = \frac{1}{1+z}$:}
\begin{gather*}
\frac{1}{1+z} = \frac{t_1^{\frac{2}{3}}}{t_0^{\frac{2}{3}}} (1+z) \implies z = \frac{t_0^{\frac{2}{3}}}{t_1^{\frac{2}{3}}} - 1
\end{gather*}
\textit{Let us rearrange for $t_1$}
\begin{gather*}
\frac{1}{1+z} = \frac{t_1^{\frac{2}{3}}}{t_0^{\frac{2}{3}}} \implies t_1^{\frac{2}{3}} = \left(t_0^{\frac{2}{3}}\right)\left(\frac{1}{1+z}\right) \implies t_1 = \left(\frac{t_0^{\frac{2}{3}}}{1+z}\right)^{\frac{3}{2}}
\end{gather*}
\subsubsection{Relativistic Hubbles Constant}
Before estimating the universe's age, let us revise our equation for Hubble Constant by applying our derivations of relativistic doppler shift. Initially, we utilized an approximation for the doppler shift at beta $\beta$ near values of zero, this is otherwise known as the classical doppler shift. $ z =\frac{v}{c}$. Therefore let us look back and rederive the doppler shift to deal with higher speeds to have a more precise Hubble constant. Which in turn allows us to probe the age of the universe in a more accurate manner.
\begin{center}
	$\boxed{
H_0 = \frac{v}{d} = \frac{ \left(\frac{z(z+2)}{z(z+2)+2}\right)c }{d} = \frac{ \left(\frac{0.1(0.1+2)}{0.1(0.1+2)+2}\right)c }{400Mpc} = 71.27 kms^{-1}/Mpc}
$
\end{center}
\subsubsection{Hubble Time}
\textbar{To find the universe's age, let us utilize time in relation to the hubbles constant}.

\textit{We previously found the Hubble constant to be $75kms^{-1}/Mpc$, thus:}
\begin{center}
	$\boxed{
	t_0 = \frac{d}{v}=\frac{1}{H_0} = \frac{3.086\times10^{19}km}{71.27kms^{-1}/Mpc} = 4.33\times10^{17}s}
	$
\end{center}
Let us rewrite this in terms of years rather than seconds.
\begin{center}
	$\boxed{
		t_0 = \frac{d}{v}=\frac{1}{H_0}= \frac{3.086\times10^{19}km}{71.27kms^{-1}/Mpc}\left(\frac{1}{(5^5\times4^4\times3^3\times2^2\times1^1)(10^{-3})(365)}\right)=13.73\times10^9yrs=13.73Gyrs}
	$
\end{center}
This approximation of the universe's age lays the foundation for us to probe the universe's age at various redshifts. Therefore when observing these redshifts, we begin to comprehend how old the light is relative to us.
\subsection{The relation between the redshift of light and the age of the universe}

\textbar{In the previous section, we alluded to the notion of an expanding universe, due to the relativistic recession speed of the Quasar. With the redshift initially provided, we may find the age of the light we observe emitted from sed Quasar:}
\begin{center}
	$\boxed{ t_1 = \left(\frac{\left(\frac{1}{H_0}\right)^{\frac{2}{3}}}{1+z_Q}\right)^{\frac{3}{2}} = \left(\frac{\left(13.73Gyrs\right)^{\frac{2}{3}}}{1+4}\right)^{\frac{3}{2}} \approx 1.23Gyrs}$
\end{center}
From this, we can convey that the age we observe in light from the Quasar is approximately 1.17 billion years old. Let us also choose arbitrary values of redshift $z$ and see how their age differs.

\textit{Choose $z = 0.5$}
\begin{center}
	$\boxed{ t_1 = \left(\frac{\left(\frac{1}{H_0}\right)^{\frac{2}{3}}}{1+z}\right)^{\frac{3}{2}} = \left(\frac{\left(13.73Gyrs\right)^{\frac{2}{3}}}{1+0.5}\right)^{\frac{3}{2}} = 7.47Gyrs}$
\end{center}

\textit{Choose $z = 1.0$}
\begin{center}
	$\boxed{ t_1 = \left(\frac{\left(\frac{1}{H_0}\right)^{\frac{2}{3}}}{1+z}\right)^{\frac{3}{2}} = \left(\frac{\left(13.73Gyrs\right)^{\frac{2}{3}}}{1+1}\right)^{\frac{3}{2}} = 4.85Gyrs}$
\end{center}

\textit{Choose $z = 3.0$}
\begin{center}
	$\boxed{ t_1 = \left(\frac{\left(\frac{1}{H_0}\right)^{\frac{2}{3}}}{1+z}\right)^{\frac{3}{2}} = \left(\frac{\left(13.73Gyrs\right)^{\frac{2}{3}}}{1+3}\right)^{\frac{3}{2}} = 1.72Gyrs}$
\end{center}

\textit{Choose $z = 10$}
\begin{center}Voro
	$\boxed{ t_1 = \left(\frac{\left(\frac{1}{H_0}\right)^{\frac{2}{3}}}{1+z}\right)^{\frac{3}{2}} = \left(\frac{\left(13.73Gyrs\right)^{\frac{2}{3}}}{1+10}\right)^{\frac{3}{2}} = 0.38Gyrs}$
\end{center}

\textit{Choose $z = 100$}
\begin{center}
	$\boxed{ t_1 = \left(\frac{\left(\frac{1}{H_0}\right)^{\frac{2}{3}}}{1+z}\right)^{\frac{3}{2}} = \left(\frac{\left(13.73Gyrs\right)^{\frac{2}{3}}}{1+100}\right)^{\frac{3}{2}} = 0.01Gyrs}$
\end{center}
Perhaps it is more convenient to make a simple graph of this relation. 

\begin{center}
\includegraphics[width=0.5\linewidth]{../../../../agevredshift}
\end{center}

\section{The Early Universe}
\subsubsection{The Mass Energy Equivalence}
\begin{gather*}
KE = W = \int_{x_1}^{x_2} Fdx
\end{gather*}
Making use of Newtons second law to rewrite the force as the rate of change of momentum. 
\begin{gather*}
	KE = W = \int_{x_1}^{x_2} \left(\frac{dp}{dt}\right)dx
\end{gather*}
\textit{We also know that momentum is defined as $p=mv$:}
\begin{gather*}
	p = m\left(\frac{dx}{dt_0}\right) = m\left(\frac{dx}{dt}\right)\left(\frac{dt}{dt_0}\right) \implies \gamma m \left(\frac{dx}{dt}\right) = \gamma mv = p
\end{gather*}
Now we have an expression for the relativistic momentum, we may differentiate with respect to t, to find the relativistic force acting on our object.
\begin{gather*}
	\frac{dp}{dt} = \frac{d}{dt}\left(\frac{mv}{\sqrt{1-\frac{v^2}{c^2}}}\right) = m\frac{d}{dt}\left(v\left(1-\frac{v^2}{c^2}\right)^{\frac{-1}{2}}\right) = m\frac{dv}{dt}\left(1-\frac{v^2}{c^2}\right)^{\frac{-1}{2}} + mv\frac{d}{dt}\left(1-\frac{v^2}{c^2}\right)^{\frac{-1}{2}}
\end{gather*}
\begin{gather*}
	\frac{dp}{dt} = m\frac{dv}{dt}\left(1-\frac{v^2}{c^2}\right)^{\frac{-1}{2}} + \frac{mv^2}{c^2}\frac{dv}{dt}\left(1-\frac{v^2}{c^2}\right)^{\frac{-3}{2}} = m\frac{dv}{dt}\left[\frac{1-\frac{v^2}{c^2}+\frac{v^2}{c^2}}{\left(1-\frac{v^2}{c^2}\right)}\right] = \gamma^3m\frac{dv}{dt}
\end{gather*}
\textit{With this, we may now insert this back into our integral to find the kinetic energy of such moving objects.}
\begin{gather*}
	KE = \int_{x_1}^{x_2} \gamma^3m\left(\frac{dv}{dt}\right)dx
\end{gather*}
\textit{Integrating with respect to $v$}
\begin{center}
	$
\boxed{KE = \int_{0}^{v} \left(\frac{mv}{\left(1-\frac{v^2}{c^2}\right)^{\frac{3}{2}}}\right)dv = \frac{mc^2}{\sqrt{1-{\frac{v^2}{c^2}}}} -mc^2 = \gamma mc^2 - mc^2 = mc^2(\gamma - 1)}
$
\end{center}
\textit{Now, let us rewrite for the total energy:}
\begin{gather*}
KE = \gamma mc^2 - mc^2 \implies KE + mc^2 = \gamma mc^2
\end{gather*}
\textit{Thus:}
\begin{center}
$\boxed{E = \gamma mc^2}
$
\end{center}
With the mass-energy equivalence, we may find the total energy content of particles with mass. For instance, let us take a baryon-antibaryon pair, the electron and positron. If they were to collide, their energy would be translated into mass. 
\subsubsection{Electron-positron pair}
\textit{Mass of electron and positron is "$9.11\times10^{-31}kg$"	(Guidry, M. W. Modern General Relativity). Thus, the energy released when these particles annihilate is:}
\begin{center}
$\boxed{E = 2mc^2 = 2(9.11\times10^{-31}kg)(3\times10^8m/s)^2 = 1.64\times10^{-13}J}
$
\end{center}
Utilizing Boltzmann's constant, we may find the temperature of such an event.
\begin{gather*}
E = kT \implies T = \frac{E}{k}
\end{gather*}
\textit{Let $k = 1.38065\times10^{-23}JK^{-1}$
}
\begin{center}
	$\boxed{T = \frac{E}{k} = \frac{1.64\times10^{-13}J}{1.38\times10^{-23}JK^{-1}}=1.90\times10^{10}K}
	$
\end{center}
\subsubsection{Proton-antiproton pair}
\textit{Mass of proton and antiproton is $1.67\times10^{-27}kg$. Thus, the energy released when these particles annihilate is:}
\begin{center}
$\boxed{E = 2mc^2 = 2(1.67\times10^{-27}kg)(3\times10^8m/s)^2 = 3.00\times10^{-10}J}
$
\end{center}
Utilizing Boltzmann's constant, we may find the temperature of such an event.

\textit{Let $k = 1.38065\times10^{-23}JK^{-1}$
}
\begin{center}
	$\boxed{T = \frac{E}{k} = \frac{3.00\times10^{-10}J}{1.38\times10^{-23}JK^{-1}}=2.17\times10^{13}K}
	$

\end{center}

The temperatures we uncovered may hint at the temperatures in the very early universe when particles came in and out of existence. These results may be somewhat may yield similar conditions found inside the particle colliders at the LHC. According to CERN, the LHC reached a temperature of five trillion Kelvin, nearly 265 times hotter than the temperature we found for the electron-positron pair. Furthermore, the annihilation between the proton-antiproton is approximately four greater than that of the highest temperature at the LHC as of 2012. 

Now we can also compare the temperatures we found to that of the core of our sun. According to NASA, the sun's core sits at a temperature of around 15 million kelvin. The temperature released from the annihilation of proton-antiproton indicates that it's 1.5 million times hotter than the sun's temperature. In addition, the temperature released from the annihilation of electron-positron is nearly 1300 times warmer than the sun's core. 

This temperature difference indicates that the current universe is far cooler than it once was long ago. The main reason we achieve such high temperatures at the LHC is that it was designed to simulate the conditions of the early universe.


\newpage
\begin{center}
	\section{References}
\end{center}

%\bibliographystyle{}

For one reason or another, the BibTex plugin is not working on my computer, will be fixed in future. Therefore I am unable to utilize a UofT-approved citation style properly. So, for now, here are the references used in the paper.


Zur elektrodynamik bewegter körper

A Einstein

Annalen der physik 4, 1905

“Sun.” NASA, NASA, solarsystem.nasa.gov/solar-system/sun/overview/. 

“Meet the Constants.” NIST, 19 Dec. 2019, www.nist.gov/si-redefinition/meet-constants.  

“A New World Record for CERN at 60 Years Old.” CERN, https://home.cern/news/news/accelerators/new-world-record-cern-60-years-old.

“Electron.” Encyclopædia Britannica, Encyclopædia Britannica, Inc., www.britannica.com/science/electron.  

“Proton.” Encyclopædia Britannica, Encyclopædia Britannica, Inc., 3 Feb. 2023, www.britannica.com/science/proton-subatomic-particle. 
\end{document}
