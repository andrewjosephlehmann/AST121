% !TeX program = lualatex

\documentclass{article}
\usepackage{tikz}
\usepackage[utf8]{inputenc}
\setlength{\parindent}{0em} %indents to paragraphs
\setlength{\parskip}{1em} %lineskips after paragraph breaks
\usepackage[margin=1.0in]{geometry} %margins
\usepackage{bbm}
\usepackage{amsmath}
\usepackage{amsthm}
\usepackage{enumerate}
\usepackage{mathtools}
\usepackage{amssymb}
\usepackage{tikz}
\usepackage{tikz-feynman}
\usepackage{amsmath}
\usepackage{physics}
\usetikzlibrary{calc}
\usepackage{feynmp}
\usepackage{feynmp-auto}
\usepackage{breqn}
\usepackage{graphicx}
\newenvironment{amatrix}[1]{%
	\left(\begin{array}{@{}*{#1}{c}|c@{}}
	}{%
	\end{array}\right)
}
\makeatletter
\let\@@span\span
\def\sp@n{\@@span\omit\advance\@multicnt\m@ne}
\makeatother

\renewcommand{\span}{...}
\newcommand{\galaxy}{\includegraphics[width=0.13in]{image}}


\title{AST121: Assignment 1}
\author{A. Lehmann}
\date{Feburary 10th 2023}

\begin{document}
	
	\maketitle

	
	\section{Dimentional Analysis}
	\textbar{With our current understanding of the universe, the cosmos was so small that gravity had similar properties to other fundamental forces. A little over a century ago, many thought the field of physics was complete. Yet, newer generations of Physicists have proven that proposition to be incorrect. However, despite the rapid progress in the 20th century, the scientific community has yet to produce a testable theory of quantum gravity. Therefore we are yet to ponder the universe at $t=0$, for now we must accept $t = 10^{-43}$ seconds after the universe's inception. 
		
	For working out the Planck time, Planck length and Planck mass. We will be referencing dimensional analysis to convert all physical quantities into dimensionless quantities, which we can easily compare and combine (e.g. we condense our understanding of length into units of meters, and for mass in units of kilograms. Combining the two would mean mass per unit length, $ML^{-1}$).
	
	\subsection{Expressing, Plancks constant $(h)$ and the speed of light $(c)$ in terms of mass $(M)$, length $(L)$ and time $(T)$}
	\textbar{Planck's constant relates the energy of a photon to its frequency and mass-energy equivalence. Thus a subset of the equations that Planck's constant will appear will describe energy and frequency. 
		
	\begin{center}
	We may set energy $E$ to be $ML^2 T^{-2}$, and frequency $f$ to be $T^{-1}$
	\end{center}
	
	By correlating the energy $(E)$ of a photon to its frequency $(f)$ multiplied by Plancks constant $(h)$, we can uncover the dimension of $h$: 
	\begin{gather*}
	E = hf \rightarrow h = Ef^{-1} \rightarrow h = (ML^2T^{-2})(T) \rightarrow h = ML^2T^{-1} 
	\end{gather*}
	
	Thus we conclude that the dimension of Planck's constant is: $h = ML^2T^{-1}$
	
	The speed of light $c$ relates to the rate at which massless particles travel in a vacuum. Thus we can use a formula for velocity which describes distance over time.
	\begin{gather*}
			v  = d/t \rightarrow v = LT^{-1}  \rightarrow c = LT^{-1}
	\end{gather*}

 }
\clearpage
\subsection{Deriving Planck length $(l_{p})$, Planck time $(t_{p})$ and Planck mass $(m_{p})$} 
\textbar{To find Planck time $t_{p}$, we need the dimension of the gravitational constant $G$, planck's constant $\hbar$ and the speed of light $c$. We have already uncovered the dimension of planck's constant and the speed of light; therefore, we must now find the gravitational constant. The gravitational constant arises in Newton's universal gravitation law and Einstein's General Relativity theory. It conveys the potency of the gravitational force between objects and determines the force a mass may experience in gravitational attraction to another. 
	
Therefore, to simplify things, let's use Newton's law of universal gravitation to find the dimension of $G$.}

\begin{gather*}
		F = \frac{G(M_{1})(M_{2})}{r^2} \rightarrow G = \frac{Fr^2}{(M_{1})(M_{2})} = \frac{(ma)r^2}{(M_{1})(M_{2})} = \frac{(M)(LT^{-2})L^2}{(M)(M)} = \frac{L^3T^{-2}}{M} = M^{-1}L^3T^{-2}
\end{gather*}
\textbar{Thus, the dimension of the gravitational constant $G$ is $M^{-1}L^3T^{-2}$. 
	
With these three fundamental constants, we can now find the fundamental unit of length, planck length $l_{p}$.}

\begin{gather*}
	L = G^{\alpha} \hbar^{\beta} c^{\gamma} \rightarrow L = \frac{L^{3\alpha}}{M^{\alpha}T^{2\alpha}} \frac{M^{\beta}T^{2\beta}}{T^{\beta}} \frac{L^{\gamma}}{T^{\gamma}}
\end{gather*}

\textit{We can write $\alpha$, $\beta$ and $\gamma$ in a matrix to determine their values for planck length $l_{p}$.}

\begin{gather*}
	\begin{bmatrix}
		2\alpha 	\ \beta \ 	\gamma \ 0 \\
		3\alpha 	\ 2\beta \ 	\gamma \ 1 \\
		-\alpha 	\ \beta \ 0 \ 0 
	\end{bmatrix}
\rightarrow R_{3}: - \alpha + \beta = 0 \rightarrow \alpha = \beta
\end{gather*}
\textit{By observing the third row, we can determine from this matrix that $\alpha = \beta$.}
\textit{Therefore rewrite the systems that represent $L$ and $T$.} 
\begin{gather*}
	3\alpha + \gamma + 1 = 5\alpha + \gamma \rightarrow 2\alpha = 1 \rightarrow \alpha = \frac{1}{2} , \beta = \frac{1}{2}	
\end{gather*}
\begin{gather*}
	2(\frac{1}{2}) + \frac{1}{2} + \gamma = 1 \rightarrow \gamma = \frac{-3}{2}
\end{gather*}

\textit{Now that we determined the value of our three unknowns, we rearrange to find the value of Planck length then we can unveil Planck time after that.}
\begin{gather*}
l_{p} = G^{1/2}\hbar^{1/2}c^{-3/2} \rightarrow l_{p} = \sqrt{\frac{G\hbar}{c^3}} = \sqrt{\frac{(6.67)(1.05)(10^{-45})}{(2.99)(10^8)}} = 1.61*10^{-35}m
\end{gather*}
\textit{To rearrange for planck time, we can use planck length divided by the speed of light.}
\begin{gather*}
	v = \frac{L}{T} \rightarrow c = \frac{l_{p}}{t_{p}} \rightarrow t_{p} = \frac{l_{p}}{c} = \frac{1.61*10^{-35}m}{2.99*10^{8}m/s} = 5.41*10^{-44}s
\end{gather*}
\textit{If we were to find the planck time similarly to planck length, we can solve in a matrix and determine that the exponents are $\alpha = 1/2, \beta = 1/2, \gamma = -5/2$.}

\clearpage


\textbar{Now that leaves Planck mass, and we can rearrange for $M$. Which has a small change to our matrix that describes the exponents which will relate to the unit of mass.}
\begin{gather*}
	M = G^{\alpha} \hbar^{\beta} c^{\gamma} \rightarrow M = \frac{L^{3\alpha}}{M^{\alpha}T^{2\alpha}} \frac{M^{\beta}T^{2\beta}}{T^{\beta}} \frac{L^{\gamma}}{T^{\gamma}} \rightarrow \begin{bmatrix}
		2\alpha 	\ \beta \ 	\gamma \ 0 \\		
		3\alpha 	\ 2\beta \ 	\gamma \ 0 \\
		-\alpha 	\ \beta \ 0 \ 1
	\end{bmatrix} = \begin{bmatrix}
	\alpha \ 0 \ 0  \ {-1/2} \\
	 0 \ \beta \ 0 \ \ {1/2}\\
	0 \ 0 \ \gamma  \ {1/2}
\end{bmatrix} 
\end{gather*}
\begin{gather*}
M = G^{\alpha} \hbar^{\beta} c^{\gamma} \rightarrow M = G^{-1/2} \hbar^{1/2} c^{1/2} = \sqrt{\frac{\hbar c}{G}} = \sqrt{\frac{(1.05*10^{-34})(2.99*10^8)}{6.67*10^{-11}}} = 2.19*10^{-8}kg
\end{gather*}
\textit{Using dimensional analysis, we found planck length $(l_{p})$, planck time $(t_{p})$ and planck mass $(m_{p})$.}
    \begin{center}
	\begin{tabular}{ |c | c | c | }
		\hline
		&&\\
		$l_{p} = 1.61*10^{-35}m$  &  $t_{p} = 5.41*10^{-44}s$ & $m_{p} = 2.19*10^{-8}kg$\\
		&&\\
		\hline
	\end{tabular} 
\end{center}

\section{Weak force interactions}

\begin{center}
	\begin{tikzpicture}
		\begin{feynman}
			\vertex (a) {\(\nu_{\tau}\)};
			\vertex [right=of a] (b);
			\vertex [above right=of b] (f1) {\(\nu_{e}\)};
			\vertex [below right=of b] (c);
			\vertex [below right=of a] (f2) {\(e^{-}\)};
			\vertex [above right=of c] (f3) {\(\tau^{-}\)};
			
			\diagram* {
				(a) -- [fermion] (b) -- [fermion] (f1),
				(b) -- [boson, edge label'=\(W^{-}\)] (c),
				(c) -- [anti fermion] (f2),
				(c) -- [fermion] (f3),
			};
		\end{feynman}
	\end{tikzpicture}
\end{center}

\textbar{The weak interaction process shown above, is known as 'electron-neutrino tau scattering.' Which a $W-$ boson is exchanged between leptons. This results in the lepton flavours changing from electron to tau and tau neutrino to electron neutrino. Typically interactions between different particles the lepton number and baryon number remains the same. Therefore we can verify that this interaction is possible, we can check the continunity of the lepton number.}
\begin{gather*}
	e^{-} + \nu_{\tau} \rightarrow \tau^{-} + \nu_{e} \leftrightsquigarrow \textit{lepton number  } 1 + 1 \rightarrow 1 + 1 \leftrightsquigarrow \textit{charge  } -1 + 0 \rightarrow -1 + 0
\end{gather*}
\begin{gather*}
	\textit{spin  } 1/2 + 1/2 \rightarrow 1/2 + 1/2
\end{gather*}
\text{We can assume that this interaction is possible since the lepton number, spin and charge is conserved.}
\begin{center}
	\begin{tikzpicture}
		\begin{feynman}
			\vertex (a) {\(e^{-}\)};
			\vertex [below right=of a] (b);
			\vertex [below left=of b] (f1) {\(e^{+}\)};
			\vertex [right=of b] (c);
			\vertex [below right=of c] (f2) {\(t\)};
			\vertex [above right=of c] (f3) {\(\nu_{\tau}\)};
			
			\diagram* {
				(a) -- [fermion] (b) -- [anti fermion] (f1),
				(b) -- [boson, edge label'=\(W^{-}\)] (c),
				(c) -- [fermion] (f2),
				(c) -- [fermion] (f3),
			};
		\end{feynman}
	\end{tikzpicture}
\end{center}
\textbar{The weak interaction process shown above is impossible since it defies energy and momentum conservation, since a baryon appears from nothing. We can verify this assumption by checking the interaction's baryon number, lepton number, charge and spin states.}
\begin{gather*}
	e^{-} + e^{+} \rightarrow t + \nu_{e} \leftrightsquigarrow \textit{baryon number	} 0 + 0 \rightarrow 1/3 + 0 \leftrightsquigarrow \textit{charge  } -1 + 1 \rightarrow 2/3 + 0
\end{gather*}
\begin{gather*}
	 \textit{spin	} 1/2 + 1/2 \rightarrow 1/2 + 1/2
\end{gather*}
\clearpage

\begin{center}
	\begin{tikzpicture}
		\begin{feynman}
			\vertex (a) {\(\nu_{e}\)};
			\vertex [right=of a] (b);
			\vertex [above right=of b] (f1) {\(e^{-}\)};
			\vertex [below right=of b] (c);
			\vertex [below right=of c] (f2) {\(u\)};
			\vertex [above right=of c] (f3) {\(d\)};
			
			\diagram* {
				(a) -- [fermion] (b) -- [fermion] (f1),
				(b) -- [boson, edge label'=\(W^{-}\)] (c),
				(c) -- [anti fermion] (f2),
				(c) -- [fermion] (f3),
			};
		\end{feynman}
	\end{tikzpicture}
\end{center}

\textbar{The weak interaction, shown above in the Feynman diagram, illustrates the process known as 'neutrino-up quark scattering,' since a down quark and electron are produced in exchange.}
\begin{gather*}
	\nu_{e} + u \rightarrow d + e^{-} \leftrightsquigarrow \textit{baryon number	} 0 + 1/3 \rightarrow 1/3 + 0 \leftrightsquigarrow \textit{charge	}  0 + 1/3 \rightarrow 1/3 + 0
\end{gather*}
\begin{gather*}
	\textit{spin	} 1/2 + 1/2 \rightarrow 1/2 + 1/2
\end{gather*}
\section{Scale anaylsis for gauging the mass of a galaxy}
\textbar{At first, it may seem overwhelming to ponder the mass of a celestial object or even a spiral galaxy. This is because so many factors are at play that comprises its mass, from supermassive black holes to empty space (which isn't empty, particles coming in and out of existence on a quantum level). Therefore it is not feasible to have an absolute measurement of the mass of a galaxy. Instead, we can estimate its mass with what contributes to it most. This mode of thinking can be referred to as scale analysis (e.i. order-of-magnitude analysis) is a technique to simplify and make rough estimates of physical quantities when modelling a system. The terms used in this model prioritize the terms with most significant magnitude and negate the with the terms with smaller magnitude. 
	
The spiral galaxy is $12.3$ million parsecs from Earth and has a bolometric luminosity (the radiant electromagnetic flux or power emitted from the galaxy per unit of time) to be $4.88 * 10^{36} Watts$. 
Since much of the mass of a given galaxy is of stars, assume the luminosity emitted from the galaxy is composed solely of stars. Thus we can use the mass-luminosity relation, which is an equation that describes the luminosity of a star in terms of its mass:	
}
\begin{gather*}
	\frac{L}{L_{\odot}} = \left(\frac{M}{M_{\odot}}\right)^a
\end{gather*}
\textbar{Since the Sun is regarded as an average star, and with our abundance of knowledge of it, we will assume that the average mass $(M_{\odot})$ luminosity $(L_{\odot})$, and it's a main-sequence star, which we assume $a = \frac{7}{2}$. Therefore we are assuming the stars that comprise the spiral galaxy are comparable to that of our Sun. 
We rearrange the mass-luminosity relation to determine the mass of our spiral galaxy:}
\begin{gather*}
	\frac{L}{L_{\odot}} = \left(\frac{M}{M_{\odot}}\right)^a = \frac{M^a}{M_{\odot}^a} \rightarrow M^a = \frac{LM_{\odot}^a}{L_{\odot}} \rightarrow M = \left(\frac{LM_{\odot}^a}{L_{\odot}}\right)^{\frac{1}{a}} 
\end{gather*}
\begin{gather*}
M_{spiralgalaxy} = \left(\frac{(4.88*10^{36}W)(1.99*10^{30}kg)^{\frac{7}{2}}}{3.86*10^{26}W}\right)^{\frac{2}{7}} = 1.53*10^{33}kg
\end{gather*}
\textit{This spiral galaxy is unreasonably light. Its mass is nearly 800 times that of our sun, which is relatively odd considering the mass of our milky way.}

\textbar{Astronomers at Nasa and ESA's Gaia measured the Milky Way's mass to be $2.98*10^{42} kg$, which is far greater than the estimate derived for our spiral galaxy. This significant deviation is likely due to us needing to consider that dust can obscure the luminosity of stars, which inhibits us from viewing the luminosity of all stars in that galaxy. In addition, we are negating any black holes, dust clouds, dark matter and other objects in a galaxy with mass emitting little light. Thus, what comprises the overwhelming majority of the mass in a given galaxy is not considered.}

\section{The Density Parameter $\Omega$ and it's effect on the universe}

\textbar{The relation between the actual density and critical density determines the universe's geometry, it's evolution and it's fate. This relation is known as the density parameter  $\Omega$.
		
	The critical density is the density that differentiates a universe that will collapse in on itself (a closed universe) from one that will expand forever (an open universe). 
	
Various existential scenarios may arise from differences in the density parameter. For instance, one may find it rational that the universe has one beginning, a point of cosmic inflation, then, with an extended interval of time, the universe becomes unstable and ends. Much of what we know about the universe's fate is the relationship between dark energy and gravity. If gravity overpowers dark energy, then the universe will collapse onto itself. Therefore this case arises when the density parameter is $\Omega < 1$.
On the other hand, if gravity cannot tame dark energy, the universe will continue to expand at a forever increasing rate—this case arises when $\Omega > 1$. Furthermore, some speculate that a 'big rip' will occur where dark energy will eventually cause the universe to expand to vast distances that it causes individual atoms to tear apart. Finally, suppose gravity and dark energy strikes a perfect balance. In that case, the universe would hold a Euclidean shape (i.e. flat), which may lead the universe will suffer heat death, in which the universe has spread its entropy evenly through the universe, where no galaxy, star or light may shine after an incomprehensible amount of time. This case arises when $\Omega = 1$. 
 }


\begin{gather*}
\Omega = \frac{\rho}{\rho_{c}} = \frac{8\pi G\rho}{3H^2}
\end{gather*}



\textbar{The density parameter has much to do with the density of matter we cannot measure, dark matter and energy. One of the many reasons is that we are comprised of baryons and leptons, responsible for around 6 percent of the universe (perhaps some undiscovered forms of matter, though unlikely). Leptons and baryons can interact with many fundamental forces, and dark matter only seems to interact with gravity, the weakest of the four fundamental forces. Dark energy accelerates the universe's expansion, breathing new life into the once-revoked cosmological constant featured in Einstein's field equations. 

Moreover, a change in the value of the Ω parameter may also impact the distribution of dark matter in the universe. Dark matter is thought to make up the majority of the matter in the universe, and its distribution may be closely linked to the value of Ω. If Ω were larger, the distribution of dark matter would be more clustered, resulting in more efficient structure formation. If Ω were smaller, the distribution of dark matter would be less clustered, resulting in less efficient structure formation.
	
In summary, the Ω parameter is a crucial concept in cosmology which describes the actual density in relation of critical density of the universe. If its value were to change, it would have significant impacts on the evolution and fate of the universe, as well as on the formation and evolution of structure and the distribution of dark matter.}

\end{document}